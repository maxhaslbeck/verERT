% Author: Andreas Lochbihler, ETH Zurich
% Version 0.2 (2014-11-19)

\makeatletter
% \newcommand{\gobble}{\@gobble}

%Umsetzung von lateinischen Kleinbuchstaben in griechische:
\def\greek@a{\alpha}
\def\greek@b{\beta}
\def\greek@c{\chi}
\def\greek@d{\delta}
\def\greek@e{\varepsilon}
\def\greek@f{\varphi}
\def\greek@g{\gamma}
\def\greek@h{\eta}
\def\greek@i{\iota}
\def\greek@k{\kappa}
\def\greek@l{\lambda}
\def\greek@m{\mu}
\def\greek@n{\nu}
\def\greek@o{\omicron}
\def\greek@p{\pi}
\def\greek@q{\theta}
\def\greek@r{\rho}
\def\greek@s{\sigma}
\def\greek@t{\tau}
\def\greek@u{\upsilon}
\def\greek@w{\omega}
\def\greek@x{\xi}
\def\greek@y{\psi}
\def\greek@z{\zeta}
\def\make@greek#1{\csname greek@#1\endcsname}
\def\@allgreek#1#2\@stopallgreek{%
  \make@greek{#1}%
  \def\@testa{}\def\@testb{#2}%
  \ifx\@testa\@testb\relax\else\@allgreek #2\@stopallgreek\fi
}
\newcommand{\greek}[1]{%
  \def\@testa{}\def\@testb{#1}
  \ifx\@testa\@testb\relax\else\@allgreek #1\@stopallgreek\fi%
}

% Formatierung fuer Formel-Konstanten
% funktioniert auch mit Hoch-/Tiefstellung, Schriftgroesse des Kontexts wird uebernommen,
% Vorschub ist 10% groesser als Schriftgroesse
\newlength{\isa@vorschub}
\ifx\beamer@version\relax
  \newcommand{\@innersyntax}[2]{%
    \def\@tempa##1{%
      \setlength{\isa@vorschub}{##1pt}%
      \mathord{\emph{\text{\fontsize{##1}{1.1\isa@vorschub}\selectfont#1#2}}}%
    }%
    \mathchoice%
      {\@tempa{\tf@size}}%
      {\@tempa{\tf@size}}%
      {\@tempa{\sf@size}}%
      {\@tempa{\ssf@size}}%
  }%
\else
  \newcommand{\@innersyntax}[2]{%
    \def\@tempa##1{%
      \setlength{\isa@vorschub}{##1pt}%
      \mathord{\text{\fontsize{##1}{1.1\isa@vorschub}\selectfont#1#2}}%
    }%
    \mathchoice%
      {\@tempa{\tf@size}}%
      {\@tempa{\tf@size}}%
      {\@tempa{\sf@size}}%
      {\@tempa{\ssf@size}}%
  }%
\fi
\newcommand{\isaconst}[1]{\textsf{\textup{#1}}}
\newcommand{\isatype}[1]{\textsf{\textup{#1}}}
\newcommand{\isacommand}[1]{\textrm{\bfseries #1}}
\newcommand{\isaminorkeyword}[1]{\textrm{#1}}
\newcommand{\isaprooftemplate}{\langle\!\langle\mathord{\text{proof}}\rangle\!\rangle}
\newcommand{\isanameprefix}[1]{\textit{#1}}
\newcommand{\isamethod}[1]{\textrm{#1}}

\def\DeclareIsaConst{\@star@or@long\Decl@reIs@Const}
\def\Decl@reIs@Const#1{%
  \expandafter\@testopt%
    \expandafter{%
    \expandafter\@newcommand%
    \expandafter{%
      \csname isa:const:#1\endcsname%
    }%
  }0%
}

\def\DeclareIsaType{\@star@or@long\Decl@reIs@Type}
\def\Decl@reIs@Type#1{%
  \expandafter\@testopt%
    \expandafter{%
    \expandafter\@newcommand%
    \expandafter{%
      \csname isa:type:#1\endcsname%
    }%
  }0%
}

\def\DeclareIs@Index#1#2#3#4#5{
  \def\testa{}\def\testb{#5}%
  \ifx\testa\testb
    \expandafter\newcommand%
    \csname isa:#1:#2\endcsname[1][]%
    {%
      \def\testa{}\def\testb{##1}\def\testc{(}\def\testd{)}%
      \ifx\testa\testb
        \index{#3@\isa{#4}}%
      \else\ifx\testb\testc
        \index{#3@\isa{#4}|(}%
      \else\ifx\testb\testd
        \index{#3@\isa{#4}|)}%
      \else
        \index{#3@\isa{#4}!##1}%
      \fi\fi\fi
    }%
  \else
    \expandafter\newcommand%
    \csname isa:#1:#2\endcsname[1][]%
    {%
      \def\testa{}\def\testb{##1}\def\testc{(}\def\testd{)}%
      \ifx\testa\testb
        \index{#3@\isa{#4}\indexxpl{#5}}%
      \else\ifx\testb\testc
        \index{#3@\isa{#4}\indexxpl{#5}|(}%
      \else\ifx\testb\testd
        \index{#3@\isa{#4}\indexxpl{#5}|)}%
      \else
        \index{#3@\isa{#4}\indexxpl{#5}!##1}%
      \fi\fi\fi
    }
  \fi
}

\def\DeclareIs@IndexSub#1#2#3#4#5{
  \expandafter\newcommand%
  \csname isa:#1:#2\endcsname[1][]%
  {%
    \def\testa{}\def\testb{##1}\def\testc{(}\def\testd{)}%
    \ifx\testa\testb
      \index{#3@\isa{#4}!#5}%
    \else\ifx\testb\testc
      \index{#3@\isa{#4}!#5|(}%
    \else\ifx\testb\testd
      \index{#3@\isa{#4}!#5|)}%
    \else
      \index{#3@\isa{#4}!#5!##1}%
    \fi\fi\fi
  }%
}

\def\DeclareIsaIndexTN{\DeclareIs@Index{indext}}
\def\DeclareIsaIndexTS{\DeclareIs@IndexSub{indext}}
\def\DeclareIsaIndexCN{\DeclareIs@Index{indexc}}
\def\DeclareIsaIndexCS{\DeclareIs@IndexSub{indexc}}

\newcommand{\DeclareIsaIndexT}{\@ifstar\DeclareIsaIndexTS\DeclareIsaIndexTN}
\newcommand{\DeclareIsaIndexC}{\@ifstar\DeclareIsaIndexCS\DeclareIsaIndexCN}

\DeclareRobustCommand{\isa}[1]{\bgroup\activateisacommands \ensuremath{#1}\egroup}
\def\inner@synt@x#1#2{%
  \expandafter\ifx\csname isa:#1:#2\endcsname\relax%
    \relax \@latex@error{Isabelle #1 #2 undefined}\@ehc%
  \else%
    \def\@tempa{\csname isa:#1:#2\endcsname}%
    \expandafter\@tempa%
  \fi%
}
\newcommand{\activateisacommands}{%
  \def\c{\inner@synt@x{const}}%
  \def\t{\inner@synt@x{type}}%
  \def\tv##1{\greek{##1}}%
  \def\hastype{\mathrel{::}}%
  \def\hassort{\mathrel{::}}%
  \def\command{\isacommand}%
  \def\minorkeyword{\isaminorkeyword}%
  \def\prooftemplate{\isaprooftemplate}%
  \def\nameprefix{\isanameprefix}%
  \def\llistOf{\overline}%
  \def\listOf{\underline}%
}
\newcommand{\isaindexc}{\inner@synt@x{indexc}}
\newcommand{\isaindext}{\inner@synt@x{indext}}
\makeatother

% Reference Isabelle theorems in LaTeX text by name without printing
\newcommand{\refthm}[1]{}

\newcommand{\isaunderscore}{\mbox{-}}

% Koinduktive Regeln mit doppelter Linie
\makeatletter
\newlength\conclusion@length
\newcommand{\oldcoinferrule}[2]{%
  \def\@premise{#1}%
  \def\@conclusion{\ensuremath{#2}}%
  \def\@testb{ }%
  \ifx\@premise\@testb%
    \relax
    \settowidth\conclusion@length{\@conclusion}%
    \def\@premise{\rule{\conclusion@length}{0cm}}%
  \fi%
  \ensuremath{{\mprset{fraction={===}}\inferrule{\@premise}{\@conclusion}}}%
}
\makeatother

\makeatletter
\newlength{\extrarowsep}
\setlength{\extrarowsep}{1.8ex}
\newcommand{\rowsep}{%
  \noalign{\ifnum0=`}\fi
  \@ifnextchar[{\@rowsep}{\@rowsep[\extrarowsep]}%
}
\def\@rowsep[#1]{%
  \hrule height #1 width 0cm \relax \ifnum0=`{\fi}%
}
\makeatother

% Tabelle fuer inferrules, die zweite Spalte fuer die Regeln nimmt so viel Platz wie moeglich,
% weniger dem Platz der ersten Spalte
\newlength{\formulacolumnwidth}%
\newenvironment{inferruletable}[1][\linewidth]{%
  % X-Spalte zentriert setzen
  \renewcommand{\tabularxcolumn}[1]{>{\setlength{\formulacolumnwidth}{##1}}c}%
  \let\oldinferrule=\inferrule%
  \let\oldcoinferrule=\coinferrule%
  \renewcommand{\inferrule}[2]{\ensuremath{\hsize=\formulacolumnwidth\oldinferrule{##1}{##2}}}%
  \renewcommand{\coinferrule}[2]{\ensuremath{\hsize=\formulacolumnwidth\oldcoinferrule{##1}{##2}}}%
  \tabularx{#1}{@{}l@{~}X@{}}%
}{
  \endtabularx%
}



\makeatletter
\def\@gatherrulesand{%
  \ifmmode $ \fi%
  \hskip 2em plus 0.5fil minus 0.5em%
  $%
}
\def\gatherrules@bindings{%
  \let\par\@gatherrulesand%
}
\newenvironment{gatherrules}{%
  \trivlist \centering \item\relax 
  \lineskiplimit=1.2em\lineskip=1.2em plus 0.2em%
  \bgroup%
  \gatherrules@bindings%
  \protect\centering%
  $%
}{$\egroup \endtrivlist}
\makeatother

\colorlet{greybackground}{black!25}
\colorlet{greyforeground}{black!50}
\newcommand{\highlight}[1]{\colorbox{greybackground}{$\!#1\!$}}
\newcommand{\highlightsmash}[1]{\vphantom{Ag}\smash{\highlight{\vphantom{X}\smash{#1}}}}

\newlength{\spacewidth}
\newcommand{\halfspace}{\settowidth{\spacewidth}{\ }\relax\hspace{0.5\spacewidth}}

\newcommand{\programfont}{\tt}
\newcommand{\programtext}[1]{\texttt{#1}}

\newcommand{\marknewfeature}{$\ast$}

